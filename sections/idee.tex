\section{Idee}

\begin{frame}
    \frametitle{Microservices Simulieren}

    \begin{columns}
        
        \begin{column}{0.5\textwidth}
            \begin{itemize}
                \item Programmatisch definieren
                \item Automaitisierte Durchführung und Auswertung von verschiedenen Szenarien und Konfigurationen
                \item Komplexere Anwedungen abbilden
                \item Berücksichtigung fachlicher Aspekte
            \end{itemize}
        \end{column}

        \begin{column}{0.5\textwidth}
            \inputminted{csharp}{code/Einleitung.cs}
        \end{column}

    \end{columns}
\end{frame}


\begin{frame}
    \frametitle{Beispiel}

    \only<1>{
        \centering
        \begin{tikzpicture}[align=center,node distance=1cm and 3cm]
            \node (caller) [circle, text width=4cm, text centered, fill=orange!70] {Aufrufer};
            \node (pol) [rectangle, fill=gray!80, below=of caller, yshift=25mm] {Retry  + Timeout};

            \node (target) [circle, text width=4cm, text centered, fill=cyan!80, right=of caller] {Ziel};
    
            \draw [arrow] (caller) -- (target);
    
    \end{tikzpicture}
    }

    \only<2>{

        \begin{columns}
            \begin{column}{0.4\textwidth}
                \begin{tikzpicture}[align=center,node distance=1cm and 1cm]
                    \node (caller) [circle, text width=2cm, text centered, fill=orange!70] {Aufrufer};
                    \node (pol) [rectangle, fill=gray!80, below=of caller, yshift=15mm] {Retry  + Timeout};
                
                    \node (target) [circle, text width=2cm, text centered, fill=cyan!80, right=of caller] {Ziel};
                
                    \draw [arrow] (caller) -- (target);
                \end{tikzpicture}
            \end{column}

            \begin{column}{0.6\textwidth}
                Hohe Anzahl an möglichen Konstellationen
                \begin{itemize}
                    \item Erreichbarkeit des Ziel-Microservices (50\%, 60\%, 70\%, 80\%, 90\%, 100\%)
                    \item Timeout Zeit in ms (1000, 1500, 2000, 2500, 3000)
                    \item Anzahl an erneuten Versuchen (0--4)
                \end{itemize}

                Allein in diesem Beispiel ergeben sich $160 = 6 * 5 * 5$ mögliche Kombinationen

                \end{column}


        \end{columns}
    }
\end{frame}

\begin{frame}
    \frametitle{Beispiel - Code}
    Problemstellung mit MICROLATION lösen
\end{frame}

\begin{frame}
    \frametitle{Beispiel - Auswertung}
\end{frame}


\begin{frame}
    \frametitle{Verbesserungspotenzial}
    \begin{itemize}
        \item Verkette Aufrufe abbilden
        \item Aggregierung der Ergebnisse, um Speicherverbrauch zu reduzieren
        \item Manipulation der Aufrufkette
        \item Auswertungsmöglichkeiten integrieren
        \item Automatische Speicherung der Ergebnisse (z.B. Datenbank)
    \end{itemize}
\end{frame}


\begin{frame}
    \frametitle{Vielen Dank für die Aufmerksamkeit}
\end{frame}