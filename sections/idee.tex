\section{Idee}

\begin{frame}
    \frametitle{Microservices Simulieren}

    \begin{columns}
        
        \begin{column}{0.5\textwidth}
            \begin{itemize}
                \item Programmatisch definieren
                \item Automaitisierte Durchführung und Auswertung von verschiedenen Szenarien und Konfigurationen
                \item Komplexere Anwedungen abbilden
                \item Berücksichtigung fachlicher Aspekte
            \end{itemize}
        \end{column}

        \begin{column}{0.5\textwidth}
            \inputminted{csharp}{code/Einleitung.cs}
        \end{column}

    \end{columns}
\end{frame}


\begin{frame}
    \frametitle{Beispiel}

    \only<1>{
        \centering
        \begin{tikzpicture}[align=center,node distance=1cm and 3cm]
            \node (caller) [circle, text width=4cm, text centered, fill=orange!70] {Aufrufer};
            \node (pol) [rectangle, fill=gray!80, below=of caller, yshift=25mm] {Retry  + Timeout};

            \node (target) [circle, text width=4cm, text centered, fill=cyan!80, right=of caller] {Ziel};
    
            \draw [arrow] (caller) -- (target);
    
    \end{tikzpicture}
    }

    \only<2>{

        \begin{columns}
            \begin{column}{0.4\textwidth}
                \begin{tikzpicture}[align=center,node distance=1cm and 1cm]
                    \node (caller) [circle, text width=2cm, text centered, fill=orange!70] {Aufrufer};
                    \node (pol) [rectangle, fill=gray!80, below=of caller, yshift=15mm] {Retry  + Timeout};
                
                    \node (target) [circle, text width=2cm, text centered, fill=cyan!80, right=of caller] {Ziel};
                
                    \draw [arrow] (caller) -- (target);
                \end{tikzpicture}
            \end{column}

            \begin{column}{0.6\textwidth}
                Drei unbekannte
                \begin{itemize}
                    \item \textbf{Erreichbarkeit} des Ziel-Microservices (50\%, 60\%, 70\%, 80\%, 90\%, 100\%)
                    \item \textbf{Timeout Zeit} in ms (1000, 1500, 2000, 2500, 3000)
                    \item \textbf{Anzahl an erneuten Versuchen} (0--4)
                \end{itemize}

                Allein in diesem Beispiel ergeben sich $160 = 6 * 5 * 5$ mögliche Kombinationen

                \end{column}


        \end{columns}
    }
\end{frame}

\begin{frame}
    \frametitle{Beispiel - Definition der Microservices}
    \inputminted{csharp}{code/Beispiel.cs}
\end{frame}


\begin{frame}
    \frametitle{Beispiel - Auswertung}


    \pgfplotstableread[col sep=semicolon]{Results.csv}\datatable

    \begin{columns}
        \begin{column}{0.5\textwidth}
            \begin{tikzpicture}
                \usetikzlibrary{plotmarks}
                \begin{axis}[
                    width=1.07\linewidth,
                    xtick=data,
                    xticklabels from table={\datatable}{Availability},
                    legend style={at={(1,0.5)},anchor=south east},
                    ylabel={Aufrufdauer in ms},
                    ylabel near ticks,
                    xlabel={Erreichbarkeit},
                    xtick pos=bottom
                ]
                
                \addplot [mark=o, blue!80] table [x expr=\coordindex, y={1000|0_T}]{\datatable};
                \addlegendentry{$1000 | 0$}
                
                \addplot [mark=o, green!80] table [x expr=\coordindex, y={1000|1_T}]{\datatable};
                \addlegendentry{$1000 | 1$}
                
                \addplot [mark=o, red!80] table [x expr=\coordindex, y={1000|2_T}]{\datatable};
                \addlegendentry{$1000 | 2$}
                
                \addplot [mark=o, cyan!80] table [x expr=\coordindex, y={1000|3_T}]{\datatable};
                \addlegendentry{$1000 | 3$}
                
                \addplot [mark=o, orange!80] table [x expr=\coordindex, y={1000|4_T}]{\datatable};
                \addlegendentry{$1000 | 4$}
                
                \end{axis}
                \end{tikzpicture}
        \end{column}

        \begin{column}{0.5\textwidth}
            \begin{tikzpicture}
                \begin{axis}[
                    width=1.07\linewidth,
                    xtick=data,
                    xticklabels from table={\datatable}{Availability},
                    legend style={at={(1,0.5)},anchor=south east},
                    ylabel={Anzahl an Fehlern},
                    ylabel near ticks,
                    xlabel={Erreichbarkeit},
                    xtick pos=bottom
                ]
                
                \addplot [mark=o, blue!80] table [x expr=\coordindex, y={1000|0_E}]{\datatable};
                \addlegendentry{$1000 | 0$}
                
                \addplot [mark=o, green!80] table [x expr=\coordindex, y={1000|1_E}]{\datatable};
                \addlegendentry{$1000 | 1$}
                
                \addplot [mark=o, red!80] table [x expr=\coordindex, y={1000|2_E}]{\datatable};
                \addlegendentry{$1000 | 2$}
                
                \addplot [mark=o, cyan!80] table [x expr=\coordindex, y={1000|3_E}]{\datatable};
                \addlegendentry{$1000 | 3$}
                
                \addplot [mark=o, orange!80] table [x expr=\coordindex, y={1000|4_E}]{\datatable};
                \addlegendentry{$1000 | 4$}
                
                \end{axis}
                \end{tikzpicture}
        \end{column}
    \end{columns}
       
    
    
       
    
    



\end{frame}


\begin{frame}
    \frametitle{Verbesserungspotenzial}
    \begin{itemize}
        \item Verkette Aufrufe abbilden
        \item Aggregierung der Ergebnisse, um Speicherverbrauch zu reduzieren
        \item Manipulation der Aufrufkette
        \item Auswertungsmöglichkeiten integrieren
        \item Automatische Speicherung der Ergebnisse (z.B. Datenbank)
    \end{itemize}
\end{frame}


\begin{frame}
    Vielen Dank für die Aufmerksamkeit!
\end{frame}